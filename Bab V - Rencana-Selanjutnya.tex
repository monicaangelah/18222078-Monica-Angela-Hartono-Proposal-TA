% ==========================================
% BAB V RENCANA SELANJUTNYA
% ==========================================
\chapter{RENCANA SELANJUTNYA}
\label{chap:rencana-selanjutnya}

\section{Rencana Implementasi}
\label{subbab:v.1}
Berdasarkan desain konsep solusi yang telah dijelaskan pada Bab \ref{chap:desain-konsep-solusi}, implementasi solusi akan dilakukan dengan pendekatan \textit{user-centered design} (UCD) dan mengacu pada standar ISO-9241-210:2010. Sebelum memasuki empat tahapan utama UCD, proses implementasi diawali dengan tahap \textit{plan the human-centered design process}. Pada tahap ini dilakukan perencanaan mengenai kebutuhan data yang akan dikumpulkan, pihak-pihak yang relevan sebagai sumber informasi, serta metode pengumpulan data yang akan digunakan. Perencanaan ini mencakup penentuan jenis informasi yang diperlukan untuk memahami perilaku dan kebutuhan pengguna. Selain itu, ditentukan target responden yang sesuai serta strategi pengumpulan data seperti wawancara, survei, ataupun observasi. Setelah tahap perencanaan, proses implementasi dilanjutkan ke empat tahapan utama UCD, yaitu:

Rencana implementasi terdiri dari empat tahapan utama, yaitu:

\begin{enumerate}
    \item Memahami dan menentukan konteks penggunaan
    \item Menentukan spesifikasi kebutuhan pengguna
    \item Pengembangan solusi desain
    \item Evaluasi solusi desain
\end{enumerate}

\vspace{1em}

Keempat tahapan tersebut saling terhubung secara iteratif sehingga hasil evaluasi pada tahap akhir dapat digunakan untuk memperbaiki desain pada tahap sebelumnya apabila diperlukan, seperti yang ditunjukkan pada Gambar \ref{gambar:interdependensi-hcd2}.

\begin{figure}[H]
    \centering
    \captionsetup{justification=centering}
    \includegraphics[width=1\textwidth]{image/Interdependensi2.jpg}
    \caption{Interdependensi Aktivitas \textit{Human-Centered Design} Menurut ISO 9241-210:2010}
    \label{gambar:interdependensi-hcd2}
\end{figure}

\subsection{Memahami dan Menentukan Konteks Penggunaan}
\label{subbab:v.1.1}

Pada tahap ini dilakukan proses untuk memahami secara menyeluruh mengenai ruang lingkup perancangan serta kondisi nyata yang dialami pengguna dalam proses penyewaan motor listrik. Kegiatan pada tahap ini mencakup identifikasi proses yang berlangsung saat ini, analisis aktivitas yang dilakukan pengguna, serta mengidentifikasi permasalahan yang muncul selama proses pencarian, pemilihan, dan penyewaan motor. Pemahaman konteks penggunaan dilakukan untuk memastikan bahwa desain interaksi yang dikembangkan sesuai dengan kebutuhan pengguna dengan mempertimbangkan aspek-aspek kognitif yang berpengaruh.

Riset pengguna dilakukan melalui pendekatan kuantitatif dan kualitatif. Pendekatan kuantitatif dilakukan dengan menyebarkan kuesioner melalui Google Form, sementara pendekatan kualitatif dilakukan melalui wawancara dengan pengguna potensial untuk mengetahui pengalaman dan kebutuhan pengguna secara lebih mendalam. Data yang diperoleh pada tahap ini digunakan untuk mengidentifikasi karakteristik pengguna, menyusun \textit{user persona}, dan menyusun \textit{user journey map} sebagai gambaran alur pengalaman pengguna. 

\subsection{Menentukan Spesifikasi Kebutuhan Pengguna}
\label{subbab:v.1.2}

Pada tahap ini dilakukan identifikasi dan penentuan kebutuhan pengguna berdasarkan hasil analisis konteks penggunaan. Kebutuhan tersebut mencakup kebutuhan fungsional, kebutuhan non-fungsional, serta kebutuhan desain interaksi yang diperlukan untuk mendukung proses penyewaan motor listrik. Identifikasi kebutuhan dilakukan dengan melihat tujuan pengguna, aktivitas yang dilakukan selama proses pencarian dan pemilihan motor, serta informasi yang dibutuhkan untuk memahami kondisi motor listrik. Selain itu, penentuan spesifikasi kebutuhan pengguna juga mempertimbangkan aspek-aspek kognitif yang berpengaruh terhadap cara pengguna menerima dan memproses informasi pada antarmuka aplikasi. Hasil dari penentuan tersebut adalah \textit{usability goals} dan \textit{user experience goals} yang ingin dicapai oleh desain, sehingga pengembangan solusi dapat dilakukan sesuai dengan kebutuhan pengguna sebenarnya.

\subsection{Menghasilkan Solusi Desain}
\label{subbab:v.1.3}

Pada tahap ini dilakukan proses pengembangan solusi desain berdasarkan kebutuhan pengguna yang telah diidentifikasi pada tahap sebelumnya. Pengembangan desain dimulai dengan menyusun arsitektur informasi untuk menentukan struktur penyajian konten yang mendukung pemahaman pengguna terhadap informasi yang berkaitan dengan motor listrik. Struktur informasi ini berfungsi sebagai dasar dalam merancang alur interaksi yang sesuai dengan tujuan dan aktivitas pengguna selama proses penyewaan. Proses pengembangan desain dilanjutkan dengan perancangan \textit{low-fidelity design} untuk menggambarkan ide awal desain antarmuka. Rancangan ini berfokus pada penentuan tata letak, alur navigasi, serta penempatan elemen-elemen penting yang mendukung proses pengambilan keputusan pengguna. Selanjutnya, rancangan dikembangkan menjadi \textit{high-fidelity design} yang menampilkan tampilan antarmuka secara lebih detail, termasuk penggunaan warna, ikon, tipografi, dan elemen visual lainnya yang disesuaikan dengan kebutuhan kognitif pengguna.

Tahap ini juga mencakup pembuatan \textit{prototype} interaktif yang digunakan sebagai representasi desain final untuk kebutuhan pengujian. \textit{Prototyoe} dirancang untuk memberikan skenario penggunaan aplikasi secara sebenarnya yang dapat menggambarkan pengalaman penggunaan. 

\subsection{Mengevaluasi Solusi Desain}
\label{subbab:v.1.4}

Pada tahap ini dilakukan evaluasi terhadap solusi desain yang telah dikembangkan untuk memastikan bahwa desain interaksi mampu membantu pengguna dalam memahami informasi dan menyelesaikan proses penyewaan motor listrik dengan baik. Evaluasi dilakukan melalui \textit{usability testing} menggunakan skenario tugas yang menggambarkan aktivitas utama pengguna. Pengujian ini bertujuan untuk menilai sejauh mana desain dapat digunakan secara efektif, efisien, dan mudah dipahami.

Evaluasi dilakukan menggunakan beberapa metrik, sebagai berikut:

\begin{enumerate} [parsep=8pt, itemsep=0pt, topsep=6pt]
    \item \textit{Success Rate/Task Completion Rate}
    
    \textit{Success rate} adalah metrik yang digunakan untuk mengukur persentase tugas yang berhasil diselesaikan oleh pengguna sesuai instruksi tanpa bantuan. Metrik ini menunjukkan sejauh mana pengguna dapat memahami alur interaksi dan menggunakan aplikasi dengan benar. Nilai \textit{success rate} yang tinggi artinya desain mampu mendukung pengguna dalam mencapai tujuan utama.
    
    \item \textit{Time on Task}
    
    \textit{Time on task} digunakan untuk mengukur waktu yang dibutuhkan pengguna dalam menyelesaikan setiap tugas selama pengujian. Metrik ini memberikan gambaran mengenai efisiensi desain, khususnya apakah pengguna dapat menemukan informasi penting dengan cepat. Waktu yang lebih singkat menunjukkan bahwa antarmuka yang ada mudah dipahami dan navigasi berjalan efektif.
    
    \item \textit{Error Rate}
    
    \textit{Error rate} adalah metrik yang digunakan untuk menghitung jumlah kesalahan yang dilakukan pengguna selama proses pengujian, baik berupa interaksi yang tidak sesuai, salah memilih elemen, maupun langkah yang tidak diperlukan. Metrik ini membantu mengetahui bagian antarmuka yang berpotensi menimbulkan kebingungan atau memberikan beban kognitif tambahan. \textit{Error rate} yang rendah menunjukkan bahwa desain mampu memberikan arahan yang jelas kepada pengguna.
    
    \item \textit{System Usability Scale} (SUS)
    
    \textit{System Usability Scale} (SUS) merupakan kuesioner yang terdiri dari sepuluh pertanyaan untuk menilai persepsi pengguna terhadap kegunaan sistem secara keseluruhan. SUS menghasilkan skor kuantitatif yang menggambarkan tingkat kenyamanan, kemudahan penggunaan, dan kepuasan pengguna terhadap aplikasi. Skor SUS digunakan sebagai indikator apakah desain telah memenuhi standar \textit{usability} yang baik.
    
    \item \textit{Single Ease Question} (SEQ)
    
    \textit{Single Ease Question} (SEQ) adalah metrik yang digunakan untuk menilai tingkat kesulitan yang dirasakan pengguna dalam menyelesaikan tugas tertentu. Pengguna diminta memberikan penilaian menggunakan skala tujuh poin, mulai dari “sangat mudah” hingga “sangat sulit”. Metrik ini membantu mengidentifikasi tugas atau bagian antarmuka mana yang dirasa paling sulit untuk digunakan sehingga dapat menjadi prioritas dalam perbaikan desain.
    
    \item \textit{Net Promoter Score} (NPS)
    
    \textit{Net Promoter Score} (NPS) adalah metrik yang digunakan untuk mengukur sejauh mana pengguna bersedia merekomendasikan aplikasi kepada orang lain berdasarkan pengalaman penggunaannya. Metrik ini digunakan untuk menilai kesan akhir pengguna dan sejauh mana desain mampu memberikan pengalaman yang positif.
\end{enumerate}

\section{Jadwal Pelaksanaan Tugas Akhir}
\label{subbab:v.2}

Pelaksanaan tugas akhir dibagi menjadi dua tahap utama, yaitu tahap pengerjaan proposal tugas akhir pada tahun 2025 dan tahap pelaksanaan tugas akhir pada tahun 2026. Pada tahap pertama, fokus kegiatan mencakup penentuan topik, pengumpulan literatur, serta penyusunan Bab \ref{chap:pendahuluan} sampai Bab \ref{chap:rencana-selanjutnya} proposal. Sementara itu, tahap kedua mencakup seluruh implementasi solusi dengan pendekatan \textit{user-centered design} (UCD). Detail jadwal pada setiap tahap ditunjukkan pada Gambar \ref{gambar:jadwal-ta-2025} dan Gambar \ref{gambar:jadwal-ta-2026}.

\begin{figure}[H]
    \centering
    \captionsetup{justification=centering}
    \includegraphics[width=1\textwidth]{image/Jadwal_Pelaksanaan_TA_2025.jpg}
    \caption{Jadwal Pelaksanaan Tugas Akhir Tahun 2025}
    \label{gambar:jadwal-ta-2025}
\end{figure}

\begin{figure}[H]
    \centering
    \captionsetup{justification=centering}
    \includegraphics[width=1\textwidth]{image/Jadwal_Pelaksanaan_TA_2026.jpg}
    \caption{Jadwal Pelaksanaan Tugas Akhir Tahun 2026}
    \label{gambar:jadwal-ta-2026}
\end{figure}

\section{Risiko dan Mitigasi}
\label{subbab:v.3}

Dalam proses pelaksanaan tugas akhir, terdapat beberapa risiko yang berpotensi memengaruhi kelancancaran pengerjaan maupun solusi yang dihasilkan. Risiko-risiko ini dapat muncul pada tahap perencanaan, pengumpulan data, pengembangan desain, maupun evaluasi desain. Karena itu, dibutuhkan mitigasi agar setiap risiko dapat dikelola dengan baik. Tabel \ref{tabel:risiko-mitigasi} berisi daftar risiko yang mungkin terjadi beserta upaya mitigasi yang dapat dilakukan. 

\begin{longtable}{|c|p{6cm}|p{6cm}|}
    
    \caption{Risiko dan Mitigasi}
    \label{tabel:risiko-mitigasi}\\
    \hline
    \textbf{Kode} & \centerline{\textbf{Risiko}} & \centerline{\textbf{Mitigasi}} \\
    \hline
    \endfirsthead

    \caption[]{Risiko dan Mitigasi (Lanjutan)}\\
    \hline
    \textbf{Kode} & \centerline{\textbf{Risiko}} & \centerline{\textbf{Mitigasi}} \\
    \hline
    \endhead

    \hline
    \multicolumn{3}{r}{\small\textit{Bersambung ke halaman berikutnya}} \\
    \endfoot
    
    \hline
    \endlastfoot
    
    RM-01 & 
    Ketidaksesuaian jadwal pengerjaan dengan rencana yang telah ditetapkan. & 
    Menyusun jadwal secara rinci untuk setiap aktivitas, memprioritaskan tugas dengan urgensi lebih tinggi, serta menyediakan waktu cadangan pada beberapa tahapan untuk mengantisipasi kondisi yang tidak terduga. \\
    \hline

    RM-02 & 
    Kesulitan dalam mengumpulkan responden penelitian yang sesuai dengan target pengguna. & 
    Memperluas jangkauan penyebaran kuesioner melalui media sosial, meminta bantuan rekomendasi responden (\textit{snowball sampling}), serta memberikan apresiasi atau \textit{reward} untuk meningkatkan partisipasi. \\
    \hline

    RM-03 & 
    Data yang didapatkan tidak valid, bias, atau tidak representatif, sehingga hasil analisis kebutuhan menjadi kurang akurat. & 
    Menggabungkan beberapa metode riset seperti survei, wawancara, dan observasi untuk meningkatkan akurasi data, serta melakukan validasi dengan melakukan wawancara tambahan jika diperlukan. \\
    \hline

    RM-04 & 
    Bias dalam memahami kebutuhan pengguna, sehingga solusi yang dikembangkan hanya menyelesaikan sebagian permasalahan pengguna. & 
    Menggunakan hasil analisis dari berbagai sumber data dan memastikan cakupan riset mencerminkan seluruh segmen pengguna yang relevan, termasuk pengguna yang belum familiar dengan motor listrik. \\
    \hline

    RM-05 & 
    Kesulitan memahami teori desain interaksi atau aspek kognitif tertentu yang menjadi dasar pengembangan solusi. & 
    Meningkatkan studi literatur, mencari referensi penelitian serupa, dan memastikan konsep-konsep yang digunakan dipahami secara menyeluruh sebelum diterapkan. \\
    \hline

    RM-06 & 
    Kerusakan teknis pada perangkat keras atau perangkat lunak, seperti laptop atau \textit{software design} tidak berfungsi optimal. & 
    Melakukan \textit{backup} data secara rutin, menyiapkan perangkat cadangan, dan memastikan seluruh perangkat lunak yang digunakan memiliki lisensi dan diperbarui secara berkala. \\
    \hline

    RM-07 & 
    Proses \textit{usability testing} tidak optimal, misalnya peserta tidak representatif, \textit{feedback} kurang mendalam, atau terdapat inkonsistensi antara iterasi pertama dan kedua. & 
    Menentukan kriteria partisipan yang sesuai dengan target pengguna, menyusun \textit{usability testing plan} dengan detail, serta melibatkan partisipan yang sama pada kedua iterasi pengujian untuk menjaga konsistensi hasil. \\
    \hline

    RM-08 & 
    Beban kognitif dan kondisi psikologis peneliti, seperti stres, kelelahan, atau kesulitan mengatur waktu dengan aktivitas lain. & 
    Menetapkan target harian yang realistis, menjaga pola istirahat, mengatur waktu kerja dengan proporsional, dan melakukan bimbingan rutin dengan dosen pembimbing. \\
    \hline

\end{longtable}