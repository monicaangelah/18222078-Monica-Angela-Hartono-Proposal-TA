% ==========================================
% BAB II STUDI LITERATUR
% ==========================================
\chapter{STUDI LITERATUR}
\label{chap:studi-literatur}

\section{Transportasi dan Permasalahan lingkungan}
\label{subbab:ii.1}

Transportasi merupakan sektor yang memegang peran penting dalam mendukung aktivitas sosial dan ekonomi masyarakat, namun peningkatan aktivitas transportasi juga membawa dampak lingkungan yang signifikan. Seiring bertambahnya jumlah kendaraan bermotor di wilayah perkotaan, kualitas udara cenderung menurun akibat meningkatnya emisi dari kendaraan berbahan bakar fosil. \textcite{guo2020impact} menunjukkan bahwa pertumbuhan infrastruktur dan aktivitas transportasi di wilayah urban berkorelasi langsung dengan memburuknya kualitas udara, terutama pada kota-kota dengan tingkat pertumbuhan kendaraan yang tinggi. Penelitian \textcite{hegazy2023evaluation} memperkuat bahwa emisi kendaraan pribadi menjadi salah satu faktor utama penurunan udara di area perkotaan, khususnya ketika jumlah kendaraan meningkat dan infrastruktur transportasi belum mampu mengakomodasi volume lalu lintas. Kondisi ini tidak hanya menimbulkan masalah lingkungan, tetapi juga berdampak pada kenyamanan dan kesehatan masyarakat.

Meningkatnya kesadaran masyarakat terhadap dampak lingkungan dari transportasi berbahan bakar fosil mendorong adopsi moda transportasi yang lebih ramah lingkungan sebagai bagian dari strategi mobilitas berkelanjutan \parencite{annisa2024sustainable}. \textcite{sari2024effect} menemukan bahwa penerapan transportasi ramah lingkungan yang disertai dengan kebijakan pemerintah mampu menurunkan tingkat polusi udara secara signifikan. Dalam konteks Indonesia, \textcite{annisa2024sustainable} juga menekankan bahwa pencapaian tujuan transportasi berkelanjutan tidak hanya memerlukan intervensi kebijakan, tetapi juga peningkatan kesadaran publik melalui edukasi dan perubahan perilaku pengguna.

\section{Kendaraan Listrik (\textit{Electric Vehicle})}
\label{subbab:ii.2}

Kendaraan listrik (\textit{electric vehicle}/EV) merupakan kendaraan yang digerakkan oleh motor listrik dan memperoleh energi dari baterai yang dapat diisi ulang \parencite{iea2023global}. Laporan \textit{Global EV Outlook 2023} yang diterbitkan oleh \textcite{iea2023global} menjelaskan bahwa kendaraan listrik menggunakan energi yang disimpan dalam baterai untuk menggerakkan motor listrik, sehingga tidak menghasilkan emisi langsung selama pengoperasiannya (\textit{zero tailpipe emissions}). Dibandingkan dengan kendaraan konvensional berbahan bakar fosil, EV menawarkan efisiensi energi lebih tinggi dan potensi pengurangan emisi yang signifikan \parencite{kurkin2024comparative}. Meski demikian, adopsi EV masih menghadapi sejumlah hambatan. \textcite{rezvani2015advances} menyebutkan bahwa aksesibilitas infrastruktur pengisian merupakan faktor kunci yang memengaruhi persepsi dan minat pengguna terhadap kendaraan listrik.

\subsection{Motor Listrik (\textit{Electric Vehicle})}
\label{subbab:ii.2.1}

Motor listrik (\textit{electric motorcycle} atau \textit{electric two-wheeler}) merupakan salah satu jenis kendaraan listrik yang menggunakan motor listrik sebagai penggerak utama dan baterai sebagai sumber energi \parencite{iea2023global}. Menurut laporan \textit{Global EV Outlook} oleh \textcite{iea2023global}, motor listrik umumnya dirancang untuk penggunaan jarak pendek hingga menengah di lingkungan perkotaan, sehingga menjadi salah satu moda transportasi yang berpotensi untuk mengurangi emisi dari sektor transportasi roda dua. Karakteristik tersebut menjadikan motor listrik relevan sebagai alternatif kendaraan harian yang unggul dalam aspek performa penggunaan di perkotaan.

Laporan \textit{Electric Motorcycle Charging Infrastructure Road Map for Indonesia} oleh \textcite{adb2022roadmap} menunjukkan bahwa peralihan dari sepeda motor berbahan bakar fosil menuju motor listrik dapat secara signifikan mengurangi polusi udara, emisi gas rumah kaca, serta ketergantungan pada bahan bakar fosil. ADB juga membahas bahwa manfaat lingkungan tersebut akan semakin optimal apabila didukung oleh ketersediaan infrastruktur pengisian daya yang memadai \parencite{adb2022roadmap}.

\subsection{Metode Pengisian Daya pada Motor Listrik}
\label{subbab:ii.2.2}

Motor listrik memerlukan pengisian daya secara berkala untuk memastikan baterai tetap memiliki energi yang cukup untuk digunakan. Berdasarkan laporan \textit{Electric Motorcycle Charging Infrastructure Road Map for Indonesia} yang diterbitkan oleh \textcite{adb2022roadmap}, sistem pengisian daya pada motor listrik di Indonesia terdiri dari dua kategori utama, yaitu \textit{battery swapping} dan \textit{battery charging infrastructure}. Pada metode \textit{battery swapping}, baterai yang hampir habis diganti dengan baterai penuh di stasiun penukaran. Sistem ini sangat diminati terutama untuk penggunaan komersial karena dapat meminimalkan waktu berhenti kendaraan.

Kategori kedua adalah \textit{battery charging infrastructure}, yaitu proses pengisian daya menggunakan \textit{charger} yang terhubung langsung ke jaringan listrik. ADB menyebutkan bahwa motor listrik dapat diisi menggunakan arus AC maupun DC, dengan teknologi pengisian dibedakan berdasarkan \textit{charging level}, \textit{charging mode}, dan \textit{charging system}. Pengisian AC terdiri dari dua tingkatan yang paling relevan untuk motor listrik, yaitu Level 1 dan Level 2. Level 1 merupakan pengisian menggunakan tegangan AC 120 volt dengan arus maksimum 16 ampere, dilakukan dengan \textit{line cord charger} yang langsung dicolokkan ke stopkontak tanpa memerlukan instalasi tambahan. Sementara Level 2, menggunakan tegangan AC 240 volt dengan arus hingga 80 ampere dan memerlukan \textit{charger} khusus yang umumnya tersedia di stasiun pengisian daya, sehingga dapat memberikan waktu pengisian yang lebih cepat dibandingkan Level 1. Selain itu, standar IEC 61851-1 membagi \textit{charging modes} menjadi Mode 1, Mode 2, dan Mode 3 untuk pengisian AC. ADB menjelaskan bahwa Mode 1 dan Mode 2 cocok untuk motor listrik, sementara Mode 3 kurang umum untuk digunakan pada kendaraan roda dua. Pengisian DC Mode 4 tidak direkomendasikan untuk roda dua karena tidak sesuai secara teknis.

\section{Layanan Rental Motor Listrik Berbasis Aplikasi}
\label{subbab:ii.3}

Layanan rental motor listrik berbasis aplikasi merupakan suatu inovasi mobilitas yang menggabungkan teknologi kendaraan listrik dengan sistem penyewaan digital. Sistem ini memungkinkan pengguna melakukan seluruh proses penyewaan melalui aplikasi, sehingga lebih praktis dibandingkan metode rental konvensional. Mekanisme operasional layanan dapat dilihat dari konsep sistem penyewaan kendaraan roda dua berbasis aplikasi yang dijelaskan oleh \textcite{bhong2023two}, yang menekankan komponen penting seperti modul pemesanan, pengelolaan ketersediaan kendaraan, verifikasi pengguna, pembayaran digital, serta fitur keamanan seperti \textit{GPS tracking}. Struktur ini relevan untuk layanan rental motor listrik karena kedua jenis layanan bergantung pada proses yang sama. Di sisi lain, penelitian terkait layanan rental \textit{e-scooter} oleh \textcite{gezmisoglu2023research} menunjukkan bahwa faktor seperti pengaruh sosial, persepsi risiko, dan kemudahan penggunaan aplikasi sangat berperan dalam membentuk minat pengguna terhadap rental kendaraan listrik.

Selain aspek sistem dan perilaku pengguna, keberhasilan layanan rental motor listrik berbasis aplikasi sangat dipengaruhi oleh faktor teknis, terutama lokasi stasiun pengisian daya, kapasitas baterai, dan estimasi jarak tempuh. Tantangan terkait infrastruktur pengisian daya di mana keterbatasan jaringan pengisian dan ketidakandalan infrastruktur menjadi salah satu hambatan utama dalam pengoperasian \textit{EV-sharing} \parencite{musida2025electric}. Kapasitas baterai menjadi faktor lain yang menentukan durasi operasional kendaraan dan memengaruhi persepsi keandalan layanan. Laporan \textit{Global EV Outlook 2023} menekankan bahwa teknologi baterai, seperti \textit{battery swapping}, memiliki pengaruh langsung terhadap efisiensi operasional, karena mempersingkat waktu pengisian dan mengurangi beban pengguna terhadap proses \textit{charging} yang lama \parencite{iea2023global}.

Sementara itu, estimasi jarak tempuh sangat berkaitan dengan fenomena \textit{range anxiety}, yaitu kekhawatiran pengguna bahwa kendaraan listrik tidak memiliki daya yang cukup untuk menyelesaikan perjalanan. \textcite{musida2025electric} menyebutkan bahwa \textit{range anxiety} merupakan salah satu faktor psikologis yang paling berpengaruh terhadap adopsi layanan rental kendaraan listrik.

\section{Desain Interaksi}
\label{subbab:ii.4}

Desain interaksi (\textit{interaction design}) berfokus pada perancangan produk digital yang memungkinkan pengguna berinteraksi secara efektif, efisien, dan menyenangkan dengan sistem digital. \textcite{Rogers2019} menjelaskan bahwa desain interaksi tidak hanya terkait tampilan visual antarmuka, tetapi juga mencakup bagaimana pengguna memahami, menavigasi, serta merasakan keseluruhan proses interaksi selama menggunakan sebuah aplikasi. Desain interaksi dapat dipahami melalui lima dimensinya sebagaimana dijelaskan oleh \textcite{teo2025what}. Dimensi pertama, \textit{words}, menekankan penggunaan kata atau label yang jelas dan bermakna agar informasi mudah dipahami tanpa membebani pengguna. Dimensi kedua, \textit{visual representations}, mencakup elemen grafis seperti ikon dan tipografi yang membantu menyampaikan informasi secara visual. Dimensi ketiga, \textit{physical objects or space}, mempertimbangkan perangkat dan konteks fisik tempat interaksi terjadi, seperti penggunaan aplikasi melalui \textit{smartphone} saat berada di ruang publik. Dimensi keempat, \textit{time}, terkait elemen yang berubah seiring waktu seperti animasi, suara, dan alur interaksi. Terakhir, dimensi kelima, \textit{behaviour}, mengacu pada bagaimana pengguna melakukan tindakan melalui antarmuka serta bagaimana sistem merespons interaksi tersebut.

Dalam desain interaksi, konseptual model berperan sebagai kerangka dasar yang menjelaskan bagaimana pengguna memahami cara kerja suatu sistem. \textcite{johnson2002conceptual} menjelaskan bahwa model konseptual merupakan representasi tingkat tinggi yang mendefinisikan objek-objek yang dikenali pengguna, atribut yang dimiliki setiap objek, operasi yang dapat dilakukan, serta hubungan antarobjek dalam suatu sistem. Dengan merancang model konseptual terlebih dahulu, desainer dapat memastikan bahwa struktur sistem selaras dengan cara pengguna memaknai tugasnya, sehingga proses belajar menjadi lebih mudah dan interaksi terasa lebih natural.

Dalam pengembangan model konseptual, metafora sering digunakan untuk membantu pengguna membangun pemahaman awal mengenai sistem melalui analogi dengan objek atau aktivitas di dunia nyata. Metafora memungkinkan pengguna mengenali fungsi atau alur interaksi tanpa harus mempelajari konsep yang sepenuhnya baru. Namun, pemilihan metafora harus dilakukan secara hati-hati agar tidak menimbulkan kesalahan pengertian.

\subsection{\textit{Human-Computer Interaction} (HCI)}
\label{subbab:ii.4.1}

\textit{Human-Computer Interaction} (HCI) merupakan bidang kajian yang berfokus pada bagaimana manusia berinteraksi dengan sistem komputasi dan bagaimana teknologi dapat dirancang agar sesuai dengan kebutuhan, kemampuan, dan keterbatasan pengguna. Seiring berkembangnya teknologi digital, HCI menghimpun beragam pendekatan dari \textit{computer science}, \textit{cognitive science}, dan \textit{human factors engineering}, yang bersama-sama membentuk dasar dalam memahami bagaimana sistem bekerja, bagaimana manusia memproses informasi, serta bagaimana aspek fisik dan perilaku pengguna memengaruhi proses interaksi.

\begin{figure}[H]
    \centering
    \captionsetup{justification=centering}
    \includegraphics[width=0.5\textwidth]{image/Bidang Multidisipliner HCI.jpg}
    \caption{Bidang Multidisipliner \textit{Human-Computer Interaction}}
    \label{gambar:multidisipliner-hci}
\end{figure}

Struktur multidisipliner tersebut tergambar pada Gambar \ref{gambar:multidisipliner-hci} yang menunjukkan posisi HCI sebagai persilangan dari ketiga bidang tersebut \parencite{carroll2014human}.

\subsection{Prinsip Desain}
\label{subbab:ii.4.2}

Dalam \textit{The Design of Everyday Things}, Norman menekankan bahwa desain yang baik harus memudahkan pengguna memahami apa yang dapat dilakukan dan bagaimana suatu sistem merespons tindakan mereka. Hal ini dicapai melalui prinsip-prinsip dasar yang membantu pengguna mengenali kemungkinan aksi, memahami konsekuensi dari setiap tindakan, serta mengurangi potensi kebingungan selama berinteraksi dengan sistem digital \parencite{norman2013design}. Prinsip-prinsip desain ini meliputi aspek seperti:

\begin{enumerate} [parsep=8pt, itemsep=0pt, topsep=6pt]
    \item \textit{Visibility} \\
    Pengguna harus dapat melihat dengan jelas fungsi atau elemen yang tersedia. Semakin terlihat suatu fungsi, semakin mudah pengguna memahami apa yang dapat dilakukan.

    \item \textit{Feedback} \\
    Sistem harus memberikan respons yang jelas dan segera setelah pengguna melakukan tindakan. Hal ini penting agar pengguna mengetahui apa yang terjadi sebagai konsekuensinya.

    \item \textit{Constraints} \\
    Pembatasan dalam desain membantu mengarahkan pengguna pada tindakan yang tepat sekaligus mencegah kesalahan. \textit{Constraints} membuat pilihan menjadi lebih jelas dan tidak membingungkan.

    \item \textit{Mapping} \\
    Terdapat hubungan yang logis dan alami antara kontrol dan hasil yang ditimbulkan. Semakin baik \textit{mapping}, semakin mudah pengguna memprediksi efek dari setiap tindakan.

    \item \textit{Consistency} \\
    Elemen dan pola interaksi yang serupa harus digunakan secara konsisten sehingga pengguna dapat mempelajari dan menggunakan sistem dengan lebih cepat tanpa perlu menyesuaikan diri dari awal.

    \item \textit{Affordance} \\
    Elemen antarmuka harus memberikan petunjuk intuitif mengenai cara menggunakannya. \textit{Affordance} memungkinkan pengguna memahami fungsi suatu objek hanya dari tampilan atau bentuknya.
\end{enumerate}

Prinsip-prinsip ini saling melengkapi dan bertujuan untuk menciptakan desain yang mudah dipahami, minim kesalahan, serta mampu mendukung interaksi yang efektif dalam berbagai konteks penggunaan perangkat digital.

\subsection{\textit{User Experience} (UX)}
\label{subbab:ii.4.3}

\textit{User experience} (UX) merujuk pada pengalaman pengguna yang tercipta ketika pengguna berinteraksi langsung dengan produk. \textcite{Rogers2019} menjelaskan bahwa UX mencakup seluruh aspek pengalaman pengguna ketika berinteraksi dengan suatu produk, termasuk perasaan, persepsi, kepuasan, serta kualitas interaksi yang dialami selama dan setelah penggunaan. UX tidak hanya berkaitan dengan kemudahan penggunaan (\textit{usability}), tetapi juga mencakup aspek emosional dan sensorik, seperti rasa nyaman ketika menggenggam perangkat, kepuasan saat melihat antarmuka yang estetis, atau kejelasan respons sistem saat pengguna melakukan suatu aksi. Desainer tidak dapat secara langsung “mendesain pengalaman”, tetapi hanya dapat merancang fitur dan karakteristik produk yang mampu memicu pengalaman tertentu. Karena itu, tujuan utama desain UX adalah menciptakan produk yang tidak hanya berfungsi dan mudah digunakan, tetapi juga memberikan kesenangan, kenyamanan, dan nilai emosional bagi pengguna dalam berbagai situasi interaksi \parencite{Rogers2019}.

\begin{enumerate} [parsep=8pt, itemsep=0pt, topsep=6pt]
    \item \textit{Effective to use (effectiveness)} \\
    \textit{Effectiveness} merujuk pada seberapa baik sistem mendukung pengguna dalam menyelesaikan tugas yang ingin dilakukan. Sistem yang efektif mendukung penyelesaian tugas tanpa hambatan, kesalahan, atau kebutuhan langkah tambahan.

    \item \textit{Efficient to use (efficiency)} \\
    \textit{Efficiency} berkaitan dengan seberapa cepat dan optimal pengguna dapat menyelesaikan tugas tanpa usaha yang berlebih. Efisiensi dipengaruhi oleh alur interaksi yang sederhana, minim langkah, serta respons sistem yang cepat.

    \item \textit{Safe to use (safety)} \\
    \textit{Safety} berkaitan dengan kemampuan sistem mencegah kesalahan dan meminimalkan konsekuensi dari kesalahan yang mungkin terjadi. Tujuannya adalah meminimalkan risiko dan memastikan pengguna tidak mengalami konsekuensi yang tidak diinginkan akibat interaksi mereka.

    \item \textit{Having good utility (utility)} \\
    \textit{Utility} berkaitan dengan sejauh mana sistem menyediakan fungsi dan fitur yang benar-benar diperlukan oleh pengguna untuk mencapai tujuannya. \textit{Utility} memastikan bahwa aplikasi dapat menyelesaikan masalah pengguna secara praktis dan sesuai konteks.

    \item \textit{Easy to learn (learnability)} \\
    \textit{Learnability} merujuk pada kemudahan bagi pengguna, terutama pengguna baru, untuk mempelajari cara menggunakan sistem tanpa memerlukan usaha yang berlebih.

    \item \textit{Easy to remember how to use (memorability)} \\
    \textit{Memorability} berkaitan dengan kemudahan bagi pengguna untuk mengingat kembali cara menggunakan sistem ketika kembali menggunakan sistem setelah tidak menggunakannya dalam beberapa waktu. Antarmuka yang konsisten, pola desain yang familiar, serta alur navigasi yang sederhana membantu pengguna mengingat langkah-langkah dalam menggunakan sistem.
\end{enumerate}

Sementara itu, \textit{user experience goals} (\textit{UX goals}) berfokus pada aspek emosional dan pengalaman subjektif pengguna yang muncul selama interaksi, seperti apakah produk terasa menyenangkan, memuaskan, menarik, membantu, atau memberikan kesan positif bagi pengguna. Jika \textit{usability goals} memastikan bahwa sistem dapat digunakan dengan baik, maka \textit{UX goals} memastikan bahwa pengguna merasa nyaman dan memperoleh pengalaman yang bermakna selama berinteraksi dengan sistem. \textit{UX goals} umumnya dibagi ke dalam dua aspek utama, yaitu aspek yang diinginkan (\textit{desirable aspects}) dan aspek yang tidak diinginkan (\textit{undesirable aspects}), seperti pada Tabel \ref{tabel:aspek-ux-goals}.

\begin{table}[H]
    \centering
    \caption{Aspek-Aspek UX \textit{Goals}}
    \label{tabel:aspek-ux-goals}
    \begin{tabular}{|l|l|l|}
        \hline
        \multicolumn{3}{|c|}{\textbf{\textit{Desirable Aspects}}} \\
        \hline
        \textit{Satisfying} & \textit{Helpful} & \textit{Fun} \\
        \hline
        \textit{Enjoyable} & \textit{Motivating} & \textit{Provocative} \\
        \hline
        \textit{Engaging} & \textit{Challenging} & \textit{Surprising} \\
        \hline
        \textit{Pleasurable} & \textit{Enhancing sociability} & \textit{Rewarding} \\
        \hline
        \textit{Exciting} & \textit{Supporting creativity} & \textit{Emotionally fulfilling} \\
        \hline
        \textit{Entertaining} & \textit{Cognitively stimulating} & \textit{Experiencing flow} \\
        \hline
        \multicolumn{3}{|c|}{\textbf{\textit{Undesirable Aspects}}} \\
        \hline
        \textit{Boring} & \textit{Unpleasant} & \textit{Creepy} \\
        \hline
        \textit{Frustrating} & \textit{Patronizing} & \textit{Intrusive} \\
        \hline
        \textit{Making one feel guilty} & \textit{Making one feel stupid} & \textit{Invasive} \\
        \hline
        \textit{Annoying} & \textit{Cutesy} & \textit{Deceptive} \\
        \hline
        \textit{Childish} & \textit{Gimmicky} & \textit{Annoying} \\
        \hline
    \end{tabular}
\end{table}

\subsection{\textit{User Interface} (UI)}
\label{subbab:ii.4.4}

\textit{User Interface} (UI) merupakan aspek desain yang berfokus pada tampilan visual dan elemen interaktif yang menjadi penghubung antara pengguna dan sistem digital \parencite{ixdf2016ui}. Menurut \textcite{ixdf2016ui}, Desain UI berfokus pada elemen visual seperti tombol, ikon, warna, tipografi, tata letak, \textit{voice-controlled interfaces}, dan interaksi berbasis gestur. Selain elemen-elemen tersebut, penggunaan warna dan kontras juga menjadi komponen penting dalam desain UI karena memengaruhi bagaimana pengguna memproses informasi dan memusatkan perhatian. Warna dapat membantu membangun hierarki visual, menonjolkan elemen yang lebih penting, serta mempermudah pengguna mengenali status sistem melalui visual, misalnya penggunaan warna hijau untuk kondisi aman atau merah untuk peringatan. Aspek seperti \textit{hue}, \textit{value}, dan \textit{saturation} berperan dalam menentukan tingkat keterbacaan dan kejelasan suatu elemen, sementara pemilihan warna yang konsisten membantu mengurangi kebingungan dan mendukung pengalaman pengguna \parencite{ixdf2016color}. Tidak hanya itu, desain UI harus mengikuti \textit{UI guidelines} terkait bagaimana elemen visual dan interaktif sebaiknya disusun, sebagai berikut:

\begin{enumerate} [parsep=8pt, itemsep=0pt, topsep=6pt]
    \item \textit{Make buttons and other common elements perform predictably so users can unconsciously use them everywhere} \\
    Tombol dan elemen interaktif lainnya harus berfungsi dengan cara yang mudah dipahami oleh pengguna.

    \item \textit{Maintain high discoverability} \\
    Ikon dan tombol harus diberi label yang jelas dan memiliki \textit{affordance} visual, seperti bayangan pada tombol, agar pengguna mudah mengetahui bahwa elemen tersebut bisa diklik.

    \item \textit{Keep interfaces simple (with only elements that help serve users' purposes) and create an ``invisible'' feel} \\
    Antarmuka harus berisi elemen yang benar-benar mendukung tujuan pengguna.

    \item \textit{Respect the user's eye and attention regarding layout} \\
    Antarmuka harus mengarahkan perhatian pengguna secara alami, seperti \textit{alignment} yang baik, hierarki visual, menggunakan warna, kontras, tipografi, dan ukuran teks yang sesuai.

    \item \textit{Minimize the number of actions required to perform tasks but focus on one chief function per page} \\
    Fokus pada fungsi utama setiap halaman dengan mengurangi langkah-langkah yang tidak perlu.

    \item \textit{Put controls near objects that users want to control} \\
    Memposisikan tombol atau kontrol di dekat elemen terkait.

    \item \textit{Keep users informed regarding system responses/actions with feedback} \\
    Sistem harus memberikan respons yang terlihat, terdengar, atau terasa agar pengguna mengetahui bahwa tindakan berhasil atau sedang diproses.

    \item \textit{Use appropriate UI design patterns to help guide users and reduce burdens} \\
    Menggunakan pola desain UI yang terkenal dan menghindari \textit{dark patterns} seperti opsi tersembunyi atau manipulatif.

    \item \textit{Maintain brand consistency} \\
    Desain harus konsisten dalam penggunaan warna, tipografi, dan ilustrasi agar \textit{brand} yang dibawakan tetap selaras.

    \item \textit{Always provide next steps which users can deduce naturally, whatever their context} \\
    Menyediakan langkah berikutnya yang dapat dipahami sesuai konteks interaksi pengguna.

    \item \textit{Tailor your UI design to the platform or device on which it's used} \\
    \textit{UI mobile} berbeda dengan \textit{UI desktop}. Desain harus mempertimbangkan ukuran layar, cara interaksi (\textit{touch vs mouse}), dan perilaku pengguna pada perangkat tersebut.

    \item \textit{Investigate UI design trends} \\
    Pendekatan seperti \textit{neomorphism}, \textit{skeuomorphism}, atau \textit{glassmorphism} dapat meningkatkan estetika, tetapi harus dipilih sesuai konteks dan kebutuhan pengguna.
\end{enumerate}

\subsection{\textit{Usability Heuristics}}
\label{subbab:ii.4.5}

\textit{Usability heuristics} merupakan suatu prinsip evaluasi antarmuka yang berfungsi untuk mengidentifikasi potensi masalah \textit{usability}, meningkatkan kenyamanan pengguna, serta memastikan bahwa interaksi berjalan lebih alami dan efisien \parencite{nng2024heuristics}. Berikut merupakan sepuluh \textit{usability heuristics} yang menjadi standar dalam penilaian antarmuka:

\begin{enumerate} [parsep=8pt, itemsep=0pt, topsep=6pt]
    \item \textit{Visibility of system status} \\
    Sistem harus selalu memberikan informasi kepada pengguna tentang apa yang sedang terjadi melalui \textit{feedback} yang tepat waktu. Ketika status sistem terlihat jelas, pengguna dapat menentukan langkah berikutnya dengan lebih percaya diri.

    \item \textit{Match between the system and the real world} \\
    Antarmuka harus menggunakan bahasa, istilah, dan konsep yang familiar bagi pengguna. Mengikuti kondisi yang ada di dunia nyata membantu pengguna memahami fungsi sistem tanpa perlu penjelasan tambahan.

    \item \textit{User control and freedom} \\
    Desain interaksi harus memberikan fleksibilitas kepada pengguna untuk membatalkan tindakan atau keluar dari proses yang tidak diinginkan (\textit{undo, redo, cancel}).

    \item \textit{Consistency and standards} \\
    Konsistensi berarti elemen yang sama harus bekerja dengan cara yang sama. Dengan mengikuti pola desain umum, pengguna tidak perlu menghafal cara baru setiap kali melihat elemen yang berbeda.

    \item \textit{Error prevention} \\
    Desain yang baik harus mencegah kesalahan sebelum terjadi. Desainer harus menghilangkan kondisi rawan kesalahan dan menyediakan mekanisme konfirmasi agar pengguna tidak melakukan tindakan yang tidak disengaja.

    \item \textit{Recognition rather than recall} \\
    Antarmuka harus mengutamakan pengenalan (\textit{recognition}) dibanding mengingat (\textit{recall}) dengan menyediakan informasi yang terlihat atau mudah diakses, sehingga mengurangi beban memori pengguna.

    \item \textit{Flexibility and efficiency of use} \\
    Antarmuka harus dapat digunakan oleh pengguna pemula maupun pengguna berpengalaman. Misalnya, pengguna berpengalaman dapat memakai \textit{shortcuts}, sementara pemula tetap bisa mengikuti langkah standar.

    \item \textit{Aesthetic and minimalist design} \\
    Antarmuka harus bebas dari elemen yang tidak relevan atau mengganggu. Desain yang minimalis dan fokus pada elemen penting meningkatkan keterbacaan dan mengurangi kebingungan.

    \item \textit{Help users recognize, diagnose, and recover from errors} \\
    Dalam desain interaksi, pesan kesalahan harus membantu pengguna memahami apa yang terjadi dan bagaimana memperbaikinya.

    \item \textit{Help and documentation} \\
    Meskipun antarmuka sebaiknya dirancang agar dapat digunakan tanpa bantuan tambahan, dokumentasi tetap diperlukan. Bantuan harus mudah dicari, ringkas, dan fokus pada langkah-langkah yang perlu dilakukan pengguna.
\end{enumerate}

\subsection{Tipe Interaksi}
\label{subbab:ii.4.6}

Menurut \textcite{Rogers2019}, tipe interaksi adalah cara utama pengguna berinteraksi dengan sebuah sistem. Pengelompokan ini membantu memahami pola aktivitas pengguna dan memilih pendekatan interaksi yang paling sesuai dengan kebutuhan pengguna. Lima tipe interaksi yang diperkenalkan adalah \textit{instructing}, \textit{conversing}, \textit{manipulating}, \textit{exploring}, dan \textit{responding} dengan detail sebagai berikut:

\begin{enumerate} [parsep=8pt, itemsep=0pt, topsep=6pt]
    \item \textit{Instructing} \\
    \textit{Instructing} merupakan jenis interaksi ketika pengguna memberikan perintah langsung kepada sistem untuk melakukan suatu tindakan. Perintah ini dapat disampaikan melalui berbagai cara, seperti menekan tombol, memilih menu, mengetik perintah, berbicara dengan sistem, atau melakukan gestur tertentu.

    \item \textit{Conversing} \\
    \textit{Conversing} merupakan jenis interaksi yang menyerupai percakapan dua arah antara pengguna dan sistem. Pengguna memberikan input dalam bentuk teks atau suara, kemudian sistem memberikan respons yang relevan, baik melalui teks ataupun suara.

    \item \textit{Manipulating} \\
    \textit{Manipulating} merujuk pada interaksi ketika pengguna memanipulasi objek digital dengan cara yang menyerupai perlakuan terhadap objek fisik, seperti men-\textit{drag}, memutar, menggeser, membuka, atau menempatkan objek. Pendekatan ini memanfaatkan pengetahuan pengguna terhadap objek-objek fisik sehingga terasa lebih mudah dipahami.

    \item \textit{Exploring} \\
    \textit{Exploring} merupakan interaksi ketika pengguna menjelajahi ruang, baik ruang virtual maupun fisik, seperti dunia 3D, \textit{augmented reality}, dan \textit{virtual reality}. Tipe interaksi ini mengandalkan pengalaman spasial pengguna untuk memahami sistem, sehingga mendukung navigasi, orientasi, dan pemahaman akan sistem.

    \item \textit{Responding} \\
    \textit{Responding} merupakan jenis interaksi ketika sistem yang memulai tindakan terlebih dahulu, kemudian pengguna memutuskan apakah ingin merespons atau mengabaikannya. Dalam desain interaksi, tantangannya adalah menentukan tingkat proaktivitas yang tepat agar cukup membantu namun tidak mengganggu.
\end{enumerate}

\subsubsection{\textit{Direct Manipulation}}
\label{subbab:ii.4.6.1}

\textit{Direct manipulation} merupakan pendekatan interaksi yang memungkinkan pengguna berinteraksi dengan sistem digital secara langsung melalui tindakan fisik yang menyerupai manipulasi objek di dunia nyata. \textcite{Rogers2019}, menyebutkan bahwa \textit{direct manipulation} dirancang agar objek digital dapat dipindahkan, dibuka, ditutup, diperbesar, atau diubah melalui aksi fisik pengguna, sehingga memberikan pengalaman interaksi yang lebih natural. \textit{Direct manipulation} didasarkan pada tiga prinsip utama, yaitu:

\begin{enumerate}
    \item Objek dan aksi yang sedang dilakukan selalu terlihat di layar, sehingga pengguna langsung tahu apa yang sedang terjadi.
    \item Setiap tindakan dapat dilakukan secara bertahap dan langsung terlihat hasilnya, sehingga pengguna bisa mengubah dan membatalkannya kapan saja.
    \item Interaksi dilakukan melalui aksi fisik seperti menggeser, mengetuk, atau melakukan gestur tertentu, bukan melalui perintah yang rumit.
\end{enumerate}

Selain ketiga prinsip tersebut, \textcite{Rogers2019} juga menjelaskan bahwa \textit{direct manipulation} memberikan berbagai manfaat bagi pengguna, antara lain:

\begin{enumerate}
    \item Membantu pengguna baru belajar dengan cepat.
    \item Meningkatkan kecepatan kerja pengguna yang sudah berpengalaman.
    \item Membantu pengguna yang jarang menggunakan aplikasi untuk tetap mengingat cara mengoperasikan sistem.
    \item Mengurangi kebutuhan munculnya pesan kesalahan.
    \item Menunjukkan kepada pengguna bagaimana setiap tindakan membawa pengguna lebih dekat ke tujuan.
    \item Mengurangi rasa cemas saat menggunakan sistem.
    \item Meningkatkan rasa percaya diri dan kontrol.
\end{enumerate}

\subsection{Aspek Kognitif}
\label{subbab:ii.4.7}

Kognisi adalah proses mental yang mencakup bagaimana manusia menerima, memproses, menyimpan, dan menggunakan informasi dalam aktivitas sehari-hari \parencite{Rogers2019}. Dalam bukunya, \textcite{Rogers2019} menyebutkan bahwa kognisi mencakup berbagai aktivitas mental seperti \textit{attention}, \textit{perception}, \textit{memory}, \textit{learning}, \textit{reading}, \textit{speaking}, \textit{listening}, \textit{problem solving}, \textit{planning}, \textit{reasoning}, dan \textit{decision making}. Kognitif merujuk pada segala sesuatu yang berkaitan dengan proses kognisi tersebut, baik kemampuan maupun keterbatasannya. Dalam desain interaksi, aspek kognitif mencerminkan bagaimana kemampuan mental pengguna memengaruhi keberhasilan mereka dalam memahami antarmuka, menyelesaikan tugas, dan mencapai tujuan.

\textcite{Rogers2019} mengatakan bahwa untuk menghasilkan desain interaksi yang baik, desainer harus memahami bagaimana proses mental pengguna memengaruhi tugas-tugas yang mereka lakukan saat berinteraksi dengan teknologi, mulai dari memahami informasi hingga mengambil keputusan. Aspek-aspek tersebut meliputi:

\begin{enumerate} [parsep=8pt, itemsep=0pt, topsep=6pt]
    \item \textit{Attention} (Perhatian) \\
    Perhatian adalah kemampuan pengguna untuk memfokuskan proses mental pada informasi tertentu sambil mengabaikan stimulus lain yang kurang relevan. Dalam konteks interaksi dengan sistem digital, perhatian sangat dipengaruhi oleh faktor seperti tampilan visual, kompleksitas informasi dan gangguan (\textit{interruptions}). \textcite{Rogers2019} menjelaskan bahwa perhatian bersifat terbatas dan mudah teralihkan, sehingga desain antarmuka harus membantu pengguna menemukan informasi penting.

    \item \textit{Perception} (Persepsi) \\
    Persepsi adalah proses ketika pengguna menangkap dan menafsirkan stimulus sensorik, terutama visual dan auditori untuk memahami elemen antarmuka. Dalam desain interaksi, persepsi visual memainkan peran dominan karena mayoritas interaksi terjadi melalui tampilan layar. \textcite{Rogers2019} menekankan bahwa persepsi dipengaruhi oleh faktor seperti warna, kontras, bentuk, ukuran, dan hierarki visual, yang secara langsung menentukan seberapa cepat dan tepat pengguna mengenali ikon, membaca teks, serta memahami struktur navigasi.

    \item \textit{Memory} (Memori) \\
    Memori adalah kemampuan pengguna untuk mengingat informasi, baik dalam jangka pendek maupun jangka panjang. \textcite{Rogers2019} menjelaskan bahwa desain antarmuka sebaiknya mengutamakan \textit{recognition rather than recall}, misalnya dengan menyediakan ikon yang familiar, label yang jelas, navigasi yang konsisten, dan elemen yang dapat digunakan tanpa perlu diingat secara berlebih. Desain yang baik harus dapat mengurangi beban memori yang mempermudah pengguna untuk menyelesaikan tugas.

    \item \textit{Learning} (Pembelajaran) \\
    Pembelajaran dalam desain interaksi berkaitan dengan bagaimana pengguna mempelajari cara menggunakan sistem dan memahami fungsi-fungsinya. Proses belajar akan menjadi lebih efektif apabila antarmuka mendukung konsistensi, menyediakan \textit{feedback} yang jelas, menggunakan metafora yang familiar, serta memungkinkan eksplorasi yang aman. \textcite{Rogers2019} mengatakan bahwa desain yang baik harus memungkinkan proses pembelajaran bertahap, sehingga pengguna tidak dibebani informasi terlalu banyak dan dapat memahami fitur baru secara bertahap seiring penggunaan sistem.

    \item \textit{Reading, Speaking, and Listening} (Membaca, Berbicara, dan Mendengarkan) \\
    Aktivitas membaca, berbicara, dan mendengarkan merupakan bagian dari proses kognitif yang melibatkan pemahaman bahasa dan interpretasi informasi. \textcite{Rogers2019} menjelaskan bahwa membaca pada layar membutuhkan perhatian terhadap faktor seperti ukuran \textit{font}, kontras, dan struktur teks untuk memastikan keterbacaan, sedangkan kemampuan berbicara dan mendengarkan memengaruhi interaksi pada sistem berbasis suara seperti asisten digital.

    \item \textit{Problem Solving, Planning, Reasoning, and Decision Making} (Pemecahan Masalah, Perencanaan, Penalaran, dan Pengambilan Keputusan) \\
    Pemecahan masalah, perencanaan, penalaran, dan pengambilan keputusan adalah proses kognitif yang terjadi ketika pengguna mencoba menentukan tindakan terbaik untuk mencapai tujuan tertentu. \textcite{Rogers2019} menjelaskan bahwa desain antarmuka harus membantu pengguna dalam membuat keputusan dengan menyediakan informasi yang jelas, mengurangi kebingungan, serta menampilkan konsekuensi pilihan dengan mudah dipahami. Oleh karena itu, sistem perlu mendukung penyederhanaan langkah, menyediakan jalur pemulihan cepat seperti \textit{undo} atau \textit{back}, dan menampilkan struktur informasi yang memudahkan perencanaan tindakan.
\end{enumerate}

Dalam \textit{Human-Computer Interaction} (HCI), \textit{cognitive frameworks} digunakan untuk menjelaskan bagaimana proses mental manusia bekerja ketika berinteraksi dengan sistem digital. \textit{Framework} ini memberikan perspektif yang membantu desainer untuk memahami bagaimana informasi diproses, disimpan, dan digunakan oleh pengguna. \textcite{Rogers2019} menjelaskan beberapa \textit{cognitive frameworks} yang dapat digunakan untuk memahami interaksi pengguna, sebagai berikut:

\begin{enumerate} [parsep=8pt, itemsep=0pt, topsep=6pt]
    \item \textit{Mental Models} \\
    \textit{Mental models} adalah gambaran di kepala pengguna tentang bagaimana sistem seharusnya bekerja. Gambaran ini terbentuk dari pengalaman pengguna sebelumnya, misalnya saat memakai aplikasi lain, menggunakan \textit{smartphone}, atau berinteraksi dengan teknologi sehari-hari. Karena itu, ketika pengguna membuka aplikasi baru, pengguna sudah memiliki ekspektasi tertentu, seperti tombol \textit{back} seharusnya akan membawa kembali ke halaman sebelumnya, ikon keranjang berarti belanja, atau ikon kaca pembesar berarti mencari. \textcite{Rogers2019} menjelaskan bahwa ketidaksesuaian antara desain dan model mental pengguna dapat menyebabkan kesalahan, kebingungan, dan meningkatkan beban kognitif. Sebaliknya, ketika desain sesuai dengan model mental, pengguna dapat mengoperasikan sistem secara intuitif.

    \item \textit{Gulf of Execution and Evaluation} \\
    Konsep \textit{gulf of execution} dan \textit{gulf of evaluation} merujuk pada bentuk kesenjangan yang terjadi antara pengguna dan sistem ketika berinteraksi. \textit{Gulf of execution} menggambarkan sejauh mana pengguna dapat memahami dan mengeksekusi untuk mencapai tujuan dalam suatu sistem, kesenjangan ini terjadi ketika pengguna tidak mengetahui cara menggunakan fungsi tertentu dalam antarmuka. Sebaliknya, \textit{gulf of evaluation} menggambarkan sejauh mana pengguna dapat memahami kondisi sistem setelah melakukan suatu tindakan, kesenjangan ini terjadi ketika sistem tidak memberikan \textit{feedback} yang jelas apakah tindakan pengguna berhasil atau tidak. \textcite{Rogers2019} menjelaskan bahwa desain yang baik harus mampu memperkecil kedua kesenjangan ini melalui \textit{affordance} yang jelas, \textit{feedback} yang informatif, serta status sistem yang mudah diartikan oleh pengguna.

    \begin{figure}[H]
        \centering
        \captionsetup{justification=centering}
        \includegraphics[width=0.6\textwidth]{image/Gulf_Execution_Evaluation.jpg}
        \caption{Ilustrasi \textit{Gulf of Execution} dan \textit{Gulf of Evaluation}}
        \label{gambar:gulfs}
    \end{figure}

    Hubungan antara kedua kesenjangan tersebut dapat dilihat pada Gambar \ref{gambar:gulfs}, yang menggambarkan alur interaksi antara pengguna dan sistem. Pada gambar tersebut, \textit{gulf of execution} berada pada sisi ketika pengguna ingin menggunakan sistem, sedangkan \textit{gulf of evaluation} muncul setelah tindakan dilakukan, ketika pengguna berusaha memahami keadaan sistem saat ini.

    \item \textit{Information Processing} \\
    \textit{Information processing} memandang manusia seperti ``mesin pemroses informasi''. Saat berinteraksi dengan aplikasi, pengguna menerima informasi dari layar (\textit{input}), memikirkan apa artinya (\textit{processing}), lalu mengambil tindakan seperti menekan tombol (\textit{output}), dan aplikasi akan merespon kembali (\textit{feedback}). \textcite{Rogers2019} menjelaskan bahwa apabila antarmuka menampilkan terlalu banyak informasi atau memberikan \textit{feedback} yang tidak jelas, proses mental pengguna menjadi lebih berat dan meningkatkan kemungkinan terjadinya kesalahan.

    \begin{figure}[H]
        \centering
        \captionsetup{justification=centering}
        \includegraphics[width=0.8\textwidth]{image/Information_Processing.jpg}
        \caption{Model Pemrosesan Informasi Manusia}
        \label{gambar:info-processing}
    \end{figure}

    Alur pemrosesan tersebut divisualisasikan pada Gambar \ref{gambar:info-processing}, yang menunjukkan tahapan \textit{encoding}, \textit{comparison}, \textit{response selection}, dan \textit{response execution} sebagai bagian dari mekanisme kognitif pengguna.

    \item \textit{Distributed Cognition} \\
    \textit{Distributed cognition} menjelaskan bahwa proses berpikir manusia tidak hanya terjadi di kepala pengguna, tetapi juga melibatkan alat, lingkungan, dan orang di sekitar. Misalnya ketika seseorang menggunakan aplikasi navigasi, pengguna melihat peta (alat), memperhatikan kondisi jalan (lingkungan), dan mungkin berdiskusi dengan teman (orang lain). \textcite{Rogers2019} menyebutkan bahwa memahami \textit{distributed cognition} membantu desainer dalam membuat desain yang bisa bekerja dengan baik di konteks dunia nyata yang melibatkan banyak sumber informasi sekaligus.

    \item \textit{External Cognition} \\
    \textit{External cognition} menjelaskan bahwa manusia sering menggunakan alat luar (misalnya catatan, peta, ikon, atau visual) untuk membantu berpikir, sehingga tidak semua informasi harus diingat di kepala. Dalam konteks aplikasi, hal ini dapat berupa ikon, daftar, \textit{progress bar}, notifikasi, atau struktur halaman yang membantu pengguna mengorganisasi informasi. \textcite{Rogers2019} memberi contoh bahwa desain yang baik harus bisa membantu pengguna dengan menyediakan representasi visual yang jelas, sehingga tidak perlu mengingat semuanya sendiri.

    \item \textit{Embodied Interaction} \\
    \textit{Embodied interaction} menjelaskan bahwa interaksi manusia dengan teknologi tidak hanya terjadi secara mental, tetapi juga melalui tubuh dan gerakan. Misalnya menyentuh layar, menggeser halaman dengan jari, memakai \textit{smartwatch}, atau menggunakan gestur di depan kamera. \textcite{Rogers2019} menjelaskan bahwa teknologi seperti \textit{touch screen}, \textit{gesture control}, dan \textit{wearable device} menunjukkan bahwa gerakan tubuh menjadi bagian penting dalam proses berinteraksi dengan sistem digital.

  \end{enumerate}

\subsection{Pendekatan Desain}
\label{subbab:ii.4.8}

Dalam proses perancangan desain interaksi, berbagai pendekatan dapat diterapkan untuk memastikan solusi yang dikembangkan sesuai dengan kebutuhan pengguna. Desain interaktif memiliki beberapa pendekatan berbeda yang dapat dipilih sesuai dengan karakteristik proyek. Pendekatan-pendekatan ini membantu menentukan fokus analisis, apakah berpusat pada aktivitas, sistem, kreativitas desainer, atau pengguna \parencite{chammas2015closer}. Pendekatan tersebut memiliki perbandingan yang diperlihatkan pada Tabel \ref{tabel:pendekatan-desain}.

\begin{longtable}{|p{0.22\linewidth}|p{0.22\linewidth}|p{0.22\linewidth}|p{0.22\linewidth}|}
    \caption{Perbandingan Pendekatan Desain Interaksi} \label{tabel:pendekatan-desain} \\
    \hline
    \textbf{Pendekatan} & \textbf{Fokus Utama} & \textbf{Pengguna} & \textbf{Desainer} \\
    \hline
    \endfirsthead
    \hline
    \textbf{Pendekatan} & \textbf{Fokus Utama} & \textbf{Pengguna} & \textbf{Desainer} \\
    \hline
    \endhead
    \hline
    \endfoot

    \textit{Activity-Centered Design} (ACD) & 
    Berfokus pada tugas dan aktivitas yang perlu dilakukan & 
    Menjadi pelaku utama aktivitas & 
    Membuat alat untuk menjalankan aktivitas \\
    \hline
    
    \textit{System Design} & 
    Berfokus pada komponen sistem yang telah dibangun & 
    Menjadi tujuan atau target dari sistem & 
    Memastikan bahwa semua bagian dari sistem sudah sesuai \\
    \hline
    
    \textit{Genius Design} & 
    Berfokus pada kemampuan dan bakat desainer untuk membuat produk & 
    Menjadi sumber validasi setelah produk dibuat & 
    Menjadi sumber inspirasi \\
    \hline
    
    \textit{User-Centered Design} (UCD) & 
    Berfokus pada tujuan dan kebutuhan pengguna & 
    Memiliki peran penting dalam desain antarmuka & 
    Memahami kebutuhan dan tujuan pengguna \\
    \hline
\end{longtable}

Desainer dapat memilih pendekatan yang paling sesuai dengan konteks dan tujuan pengembangan. Setiap pendekatan memiliki karakteristik yang unik dan relevan untuk konteks pengguna yang berbeda-beda. Berdasarkan perbandingan empat pendekatan desain, penelitian ini paling relevan menggunakan \textit{user-centered design} (UCD) karena secara langsung menempatkan kebutuhan, tujuan, dan pengalaman pengguna sebagai pusat proses perancangan.

\subsubsection{\textit{User-Centered Design} (UCD)}
\label{subbab:ii.4.8.1}

\textit{User-Centered Design} (UCD) atau \textit{human-centered design} menurut ISO 9241-210:2010 merupakan pendekatan desain dan pengembangan sistem yang menempatkan pengguna, kebutuhan, dan konteks penggunaan sebagai pusat dari seluruh proses desain \parencite{iso2010}. Menurut \textcite{ixdf2016ucd}, UCD merupakan proses desain iteratif dimana desainer secara aktif melibatkan pengguna dalam setiap tahap, mulai dari pemahaman konteks, identifikasi kebutuhan, pengembangan solusi, hingga evaluasi.

ISO 9241-210:2010 juga menjelaskan berbagai alasan penting mengadopsi pendekatan UCD, yaitu:
\begin{enumerate}
    \item Meningkatkan produktivitas pengguna dan efisiensi operasional organisasi.
    \item Memudahkan sistem untuk dipahami dan digunakan.
    \item Meningkatkan kegunaan untuk pengguna dengan kemampuan yang beragam sehingga memperluas aksesibilitas.
    \item Meningkatkan \textit{user experience} secara keseluruhan.
    \item Mengurangi ketidaknyamanan dan stres yang mungkin dialami pengguna saat berinteraksi dengan sistem.
    \item Memberikan \textit{competitive advantage}, misalnya melalui peningkatan citra merek.
    \item Mendukung tujuan berkelanjutan.
\end{enumerate}

Selain itu, ISO 9241-210:2010 memaparkan prinsip-prinsip dalam perancangan desain interaksi yang berpusat pada pengguna, yaitu:
\begin{enumerate}
    \item Desain didasarkan pada pemahaman yang jelas mengenai pengguna, tugas, dan lingkungan.
    \item Pengguna terlibat secara aktif sepanjang proses desain dan pengembangan.
    \item Desain disempurnakan melalui evaluasi yang berpusat pada pengguna.
    \item Proses bersifat iteratif.
    \item Desain mempertimbangkan keseluruhan pengalaman pengguna (\textit{user experience}).
    \item Tim desain melibatkan keahlian multidisiplin.
\end{enumerate}

Dalam pendekatan \textit{User-Centered Design} (UCD), ISO 9241-210:2010 menjelaskan bahwa proses desain tidak hanya terdiri dari empat aktivitas inti, tetapi juga diawali oleh tahap perencanaan yang bertujuan untuk menentukan bagaimana keseluruhan proses UCD akan dijalankan. Tahap awal ini merupakan \textit{plan the human-centered design process}, yang berfungsi sebagai fondasi untuk memastikan bahwa aktivitas desain selanjutnya dilakukan secara terarah dan sesuai dengan kebutuhan pengguna. Pada tahap perencanaan, dilakukan penetapan jenis informasi yang perlu dikumpulkan, menentukan karakteristik pengguna yang relevan, serta menentukan metode pengumpulan data yang tepat seperti wawancara, survei, observasi, atau analisis dokumen. \textcite{Rogers2019} menjelaskan bahwa perencanaan ini penting untuk memastikan bahwa proses \textit{data gathering} berlangsung secara terstruktur dan mendapatkan hasil yang valid terkait konteks penggunaan. Selain itu, tahap ini membantu dalam mengidentifikasi hipotesis awal yang berasal dari data sekunder sehingga dapat divalidasi melalui data primer dari pengguna. Setelah tahap perencanaan selesai dilakukan, dilanjutkan dengan keempat aktivitas utama dalam proses UCD. Alur keempat aktivitas tersebut dijelaskan lebih lanjut pada Gambar \ref{gambar:interdependensi-hcd}, sebagai berikut:


\begin{figure}[H]
    \centering
    \captionsetup{justification=centering}
    \includegraphics[width=0.7\textwidth]{image/Interdependensi Aktivitas Human-Centered Design.jpg}
    \caption{Interdependensi Aktivitas \textit{Human-Centered Design} Menurut ISO 9241-210:2010}
    \label{gambar:interdependensi-hcd}
\end{figure}

\begin{enumerate} [parsep=8pt, itemsep=0pt, topsep=6pt]
    \item \textit{Understanding and specifying the context of use} \\
    Tahap pertama berfokus pada pemahaman mengenai bagaimana sistem akan digunakan dalam situasi nyata. Tahap ini meliputi identifikasi siapa pengguna yang terlibat, apa tujuan mereka, tugas apa yang harus dilakukan, serta kondisi lingkungan fisik, sosial, dan teknis yang memengaruhi interaksi. Pemahaman konteks penggunaan sangat penting untuk menghindari kesalahan pemahaman mengenai kebutuhan pengguna dan memastikan bahwa desain benar-benar mencerminkan realitas penggunaan.

    \item \textit{Specify the user requirements} \\
    Berdasarkan informasi dari konteks penggunaan, kebutuhan pengguna kemudian dirumuskan menjadi \textit{user requirements} yang lebih terstruktur. Kebutuhan ini mencakup kebutuhan fungsional terkait apa yang harus bisa dilakukan sistem, maupun kebutuhan nonfungsional seperti kemudahan penggunaan, kenyamanan, dan batasan lingkungan.

    \item \textit{Producing design solutions to meet user requirements} \\
    Pada tahap ini, mulai dilakukan pembuatan solusi berdasarkan kebutuhan pengguna yang telah dirumuskan. Bentuknya dapat berupa sketsa awal, \textit{wireframe}, \textit{flowchart}, \textit{low-fidelity prototype}, hingga \textit{high-fidelity prototype} dengan tujuan untuk memenuhi kebutuhan pengguna sekaligus mempertimbangkan pengalaman pengguna.

    \item \textit{Evaluate the design against requirements} \\
    Tahap evaluasi dilakukan untuk menguji sejauh mana solusi desain memenuhi kebutuhan pengguna yang telah ditetapkan. Evaluasi dapat dilakukan melalui berbagai metode seperti \textit{usability testing}, \textit{heuristic evaluation}, \textit{walkthrough}, maupun observasi secara langsung. \textit{Feedback} dari pengguna menjadi dasar untuk menilai apakah desain memerlukan perbaikan atau tidak.
\end{enumerate}

\subsection{Evaluasi Desain}
\label{subbab:ii.4.9}

Evaluasi desain merupakan proses untuk menentukan sejauh mana desain antarmuka memenuhi \textit{usability goals} dan \textit{user experience goals}. Evaluasi merupakan bagian penting dari proses desain interaktif untuk memastikan bahwa produk layak digunakan sesuai kebutuhan pengguna \parencite{Rogers2019}.

\subsubsection{\textit{Usability Testing}}
\label{subbab:ii.4.9.1}

\textit{Usability testing} merupakan metode evaluasi yang dilakukan untuk mengetahui sejauh mana pengguna dapat menggunakan suatu antarmuka secara efektif, efisien, dan memuaskan. Pengujian ini melibatkan pengguna yang mewakili target sistem dan dilakukan dengan cara mengamati bagaimana pengguna menyelesaikan tugas tertentu pada sistem yang telah dikembangkan. \textcite{Rogers2019} menjelaskan bahwa \textit{usability testing} bertujuan untuk mengidentifikasi masalah penggunaan, mengamati perilaku pengguna, dan memastikan bahwa desain yang dibuat benar-benar mendukung kebutuhan pengguna sebenarnya.

Evaluasi dapat diperkuat menggunakan kerangka \textit{5Es Usability}, yaitu \textit{effective}, \textit{efficient}, \textit{engaging}, \textit{error tolerant}, dan \textit{easy to learn}. Kerangka ini diperkenalkan oleh \textcite{quesenbery2004balancing} sebagai cara untuk memahami apa saja yang memengaruhi kualitas \textit{usability} dalam konteks penggunaan nyata:

\begin{enumerate}
    \item \textit{Effective}, mengukur sejauh mana pengguna dapat mencapai tujuan dengan tepat.
    \item \textit{Efficient}, mengukur kecepatan dan usaha yang dibutuhkan.
    \item \textit{Engaging}, mengukur sejauh mana antarmuka nyaman dan menyenangkan untuk digunakan.
    \item \textit{Error tolerant}, mengukur bagaimana sistem mencegah dan membantu pemulihan dari kesalahan.
    \item \textit{Easy to learn}, mengukur kemudahan pengguna dalam mempelajari dan memahami sistem, baik pada penggunaan pertama maupun penggunaan berulang.
\end{enumerate}

\subsubsection{Metrik \textit{Usability Testing}}
\label{subbab:ii.4.9.2}

Dalam \textit{usability testing}, terdapat beberapa metrik kuantitatif yang biasanya digunakan untuk mengukur kualitas interaksi pengguna secara objektif. Pemilihan metrik disesuaikan dengan target \textit{usability goals} dan \textit{UX goals} yang ingin dicapai. \textcite{nielsen2001metrics} menjelaskan bahwa metrik dasar \textit{usability} mencakup empat komponen utama, sebagai berikut:

\begin{enumerate} [parsep=8pt, itemsep=0pt, topsep=6pt]
    \item \textit{Success Rate} \\
    \textit{Success rate} atau \textit{task completion rate} (TCR) merupakan metrik dalam \textit{usability testing} yang digunakan untuk mengukur sejauh mana pengguna mampu menyelesaikan suatu tugas sesuai dengan tujuan yang ditetapkan. Metrik ini menunjukkan proporsi pengguna yang berhasil menyelesaikan tugas tanpa kesalahan yang menghambat atau tanpa bantuan. \textcite{nielsen2001metrics} menjelaskan bahwa \textit{success rate} merupakan indikator paling mendasar dalam evaluasi \textit{usability}, karena secara langsung menunjukkan efektivitas desain dalam mendukung penyelesaian tugas oleh pengguna.

    \textit{Task completion rate} didefinisikan sebagai persentase tugas yang berhasil diselesaikan dibandingkan total tugas yang diberikan. Setelah dilakukan perhitungan, TCR pada kisaran 90-100\% menunjukkan performa yang sangat baik, dimana pengguna mampu menyelesaikan tugas dengan lancar, artinya desain sudah efisien dan mendukung efektivitas penggunaan. Kisaran 70-89\% menunjukkan performa yang baik namun masih terdapat beberapa hambatan yang perlu diperbaiki untuk meningkatkan konsistensi pengalaman pengguna. Kisaran 50-69\% menunjukkan performa rata-rata dan adanya masalah \textit{usability} yang cukup signifikan dan dapat mengganggu penyelesaian tugas, sehingga diperlukan evaluasi lebih lanjut. Sementara TCR $<$50\% menunjukkan performa yang buruk, menandakan adanya permasalahan pada desain antarmuka yang harus segera diperbaiki \textcite{askarov2024task}.

    \item \textit{Time on Task} \\
    \textit{Time on task} merupakan metrik yang digunakan untuk mengukur lamanya waktu yang dibutuhkan pengguna dalam menyelesaikan suatu tugas tertentu pada sistem. \textcite{ixdf2023kpi} menjelaskan bahwa waktu penyelesaian tugas yang lebih lama seringkali menunjukkan bahwa pengguna mengalami kesulitan untuk memahami alur interaksi atau menemukan informasi yang dibutuhkan, sehingga desain perlu dievaluasi ulang. 

    Pengukuran \textit{time on task} dilakukan dengan mencatat waktu pada setiap fitur, serta menghitung total durasi dari awal hingga tugas selesai. Analisis \textit{time on task} biasanya dilaporkan dalam bentuk rata-rata, untuk sampel kecil perhitungan dilakukan menggunakan \textit{geometric mean} agar tidak terdistorsi oleh \textit{outlier}, sedangkan untuk sampel yang lebih besar perhitungan dilakukan menggunakan median.

    \item \textit{Error Rate} \\
    \textit{Error rate} merupakan metrik yang digunakan untuk mengukur jumlah dan jenis kesalahan yang dilakukan pengguna selama menyelesaikan tugas pada suatu antarmuka. \textcite{nielsen2001metrics} menjelaskan bahwa \textit{error rate} tidak hanya mencatat jumlah kesalahan, tetapi juga memperhatikan apakah kesalahan tersebut bersifat minor, mayor, atau kritis, dimana kesalahan kritis dapat menyebabkan pengguna gagal menyelesaikan tugas secara keseluruhan. Pengukuran \textit{error rate} dilakukan dengan mencatat setiap kesalahan yang muncul selama pengujian, termasuk salah klik, salah \textit{input}, kebingungan navigasi, atau tindakan yang tidak menghasilkan respons yang diharapkan.

    \item \textit{Users' Subjective Satisfaction} \\
    \textit{Users' subjective satisfaction} merupakan metrik yang digunakan untuk menilai persepsi, kenyamanan, serta pengalaman pengguna terhadap antarmuka setelah menyelesaikan tugas dalam \textit{usability testing}. Metrik ini dapat diukur dengan berbagai macam pengukuran, seperti , \textit{System Usability Scale} (SUS), \textit{Single Ease Question} (SEQ), dan \textit{Net Promoter Score} (NPS). 
    \begin{enumerate}
        \item \textit{System Usability Scale} (SUS) \\
        \textit{System Usability Scale} (SUS) merupakan skala evaluasi yang terdiri dari sepuluh pertanyaan dengan skala likert lima poin, dirancang untuk memberikan penilaian cepat dan sederhana terhadap persepsi \textit{usability} suatu produk \textcite{bangor2009sus}. Sepuluh pertanyaan tersebut terdiri dari lima pertanyaan positif dan lima pertanyaan negatif. Semua item pada skala harus diisi. Pengguna dapat memilih skala likert dari nilai 1 (sangat tidak setuju/\textit{strongly disagree}) hingga 5 (sangat setuju/\textit{strongly agree}) untuk menjawab semua pertanyaan.

        \vspace{1em}

        Perhitungan SUS dilakukan berdasarkan respons pengguna terhadap sepuluh pertanyaan. Setiap item pada SUS diberi nilai berbeda tergantung apakah pertanyaan tersebut bersifat positif atau negatif. Menurut \textcite{bangor2009sus}, langkah perhitungannya adalah untuk item positif yang berada pada nomor ganjil (1, 3, 5, 7, dan 9), skor kontribusi dihitung dengan mengurangkan nilai jawaban sebesar satu, sedangkan untuk item negatif pada nomor genap (2, 4, 6, 8, 10), skor kontribusi diperoleh dengan mengurangkan nilai jawaban dari lima. Setelah seluruh skor kontribusi dihitung, nilai-nilai tersebut dijumlahkan sehingga menghasilkan total skor dengan rentang 0 hingga 40. Total tersebut kemudian dikalikan dengan 2,5 sehingga menghasilkan skor akhir SUS dalam rentang 0 hingga 100, yang merepresentasikan tingkat \textit{usability} suatu produk.
        
        \begin{figure}[H]
            \centering
            \captionsetup{justification=centering}
            \includegraphics[width=0.8\textwidth]{image/SUS-Scale.jpg}
            \caption{Skala Penilaian \textit{System Usability Scale} (SUS)}
            \label{gambar:sus-scale}
        \end{figure}

        Skor SUS yang lebih tinggi menunjukkan persepsi \textit{usability} yang lebih baik. Interpretasi terhadap skor SUS juga dapat dilihat melalui pemetaan kategori seperti pada Gambar \ref{gambar:sus-scale}, dimana skor $<$50 termasuk kategori \textit{Non Acceptable}, skor 50-70 berada pada kategori \textit{Marginal}, dan skor $>$70 berada dalam kategori \textit{Acceptable}.

        \item \textit{Single Ease Question} (SEQ) \\
        \textit{Single Ease Question} (SEQ) merupakan penilaian \textit{post task} yang digunakan untuk mengukur persepsi pengguna terhadap tingkat kemudahan dalam menyelesaikan suatu tugas yang baru saja dilakukan \textcite{laubheimer2018measuring}. Pengguna menilai tingkat kesulitan tugas yang baru saja dilakukan menggunakan skala likert 7 poin mulai dari \textit{very difficult} hingga \textit{very easy}, seperti pada Gambar \ref{gambar:seq-scale}.
        
        \begin{figure}[H]
            \centering
            \captionsetup{justification=centering}
            \includegraphics[width=0.8\textwidth]{image/SEQ-Scale.jpg}
            \caption{Skala \textit{Single Ease Question} (SEQ)}
            \label{gambar:seq-scale}
        \end{figure}

        SEQ efektif digunakan karena dapat menunjukkan bagian antarmuka atau tugas yang menimbulkan kesulitan bagi pengguna. Oleh karena itu, SEQ sangat membantu dalam membedakan tugas yang dirasa lebih mudah maupun lebih sulit dalam suatu sistem.

        \item \textit{Net Promoter Score} (NPS) \\
        \textit{Net Promoter Score} (NPS) merupakan metrik yang digunakan untuk mengukur tingkat loyalitas pengguna dan kecenderungan untuk merekomendasikan sebuah produk atau layanan ke orang lain. Dalam pengukurannya, pengguna diminta untuk menjawab pertanyaan seperti ``\textit{How likely are you to recommend this website/product/service to a friend or relative?}'' dengan skala 0-10. Berdasarkan respons tersebut, pengguna diklasifikasikan menjadi tiga kelompok, yaitu \textit{promoters} (nilai 9-10), \textit{passives} (nilai 7-8), dan \textit{detractors} (nilai 0-6).

        \vspace{1em}

        NPS kemudian dihitung dengan mengurangi persentase \textit{detractors} dari persentase \textit{promoters}, sehingga menghasilkan skor antara -100\% hingga +100\%. Skor -100\% menunjukkan bahwa seluruh responden termasuk ke dalam kategori \textit{detractors}, sedangkan skor +100\% menunjukkan bahwa seluruh responden merupakan \textit{promoters}. Skor yang bernilai positif menunjukkan bahwa jumlah \textit{promoters} lebih besar daripada \textit{detractors}, sedangkan skor negatif menunjukkan tingkat loyalitas pengguna yang rendah karena jumlah \textit{detractors} lebih tinggi daripada \textit{promoters} \textcite{fessenden2024nps}.

    \end{enumerate}
\end{enumerate}

\section{Penelitian Terkait}
\label{subbab:ii.5}

\subsection{\textit{A Survey on Measuring Cognitive Workload in Human-Computer Interaction}}
\label{subbab:ii.5.1}

\textcite{kosch2023survey} melakukan survei mengenai berbagai pendekatan dalam mengukur \textit{cognitive workload} dalam konteks \textit{Human-Computer Interaction} (HCI). Penelitian tersebut menjelaskan bagaimana meningkatnya kompleksitas sistem digital membuat beban kognitif semakin penting diperhatikan dalam proses perancangan maupun evaluasi antarmuka. Beban mental yang terlalu tinggi dapat mengurangi kinerja pengguna, mengganggu fokus, dan menurunkan kualitas pengalaman pengguna secara keseluruhan. Dalam penelitiannya, Kosch et al. menggunakan berbagai pendekatan pengukuran mulai dari metode subjektif, seperti NASA-TLX, DSSQ, dan kuesioner lain, hingga metode objektif yang memanfaatkan indikator fisiologis seperti \textit{pupil dilation}, \textit{eye tracking}, \textit{heart rate variability}, \textit{electrodermal activity} (EDA), EEG, dan fNIRS. Masing-masing metode memiliki karakteristik, kelebihan, dan keterbatasan tersendiri, sehingga pemilihannya perlu disesuaikan dengan tujuan penelitian dan jenis tugas yang diuji. \textcite{kosch2023survey} juga mengelompokkan teknik pengukuran ke dalam tiga kategori utama, yaitu \textit{self report measures}, \textit{behavior based measures}, dan \textit{physiological measures}. Penelitian ini menyimpulkan bahwa tidak ada satu metode pengukuran yang unggul dalam semua situasi, sehingga kombinasi beberapa pendekatan diperlukan untuk memperoleh gambaran yang lebih jelas.

\subsection{RENTALIN APPS: \textit{Development of a Motorbike Rental Application Using Design Thinking for a Digital Rental Solution}}
\label{subbab:ii.5.2}

\textcite{salsabilah2024rentalin} mengembangkan aplikasi RENTALIN sebagai solusi untuk mengatasi berbagai permasalahan yang dihadapi industri rental motor di Indonesia. Penelitian ini berawal dari permasalahan awal bahwa pelaku usaha rental motor menghadapi sejumlah tantangan serius seperti risiko pencurian, penipuan, kesulitan verifikasi identitas pelanggan, hingga keterbatasan dalam menjangkau pasar yang lebih luas. Dalam penelitian ini, para peneliti melakukan wawancara dan survei terhadap pemilik rental motor di berbagai wilayah, hasilnya menunjukkan bahwa 92,3\% pemilik membutuhkan platform \textit{online}, 100\% memerlukan sistem pelacakan motor, dan 100\% responden pernah mengalami masalah keamanan seperti pencurian kendaraan. Selain itu, survei terhadap penyewa motor menunjukkan bahwa 100\% responden mengalami kesulitan dalam menemukan layanan rental yang terpercaya, terutama di wilayah nonwisata, serta menghadapi risiko penipuan, dan kurangnya informasi mengenai ketersediaan kendaraan \parencite{salsabilah2024rentalin}.

Penelitian ini menggunakan pendekatan \textit{Design Thinking} dengan lima tahapan, \textit{Empathize}, \textit{Define}, \textit{Ideate}, \textit{Prototype}, dan \textit{Test}. Pada tahap \textit{empathize}, pengumpulan data dilakukan melalui wawancara pemilik rental di Depok dan survei pelanggan. Hasil analisis tersebut kemudian dipetakan ke dalam tahap \textit{define}, dan dilanjutkan dengan pengembangan ide pada tahap \textit{ideate} untuk menghasilkan solusi yang relevan. Tahap \textit{prototype} menghasilkan rancangan awal aplikasi RENTALIN dengan fitur-fitur seperti pelacakan GPS, sistem verifikasi data, serta antarmuka yang dirancang agar mudah digunakan. Pada tahap \textit{test}, prototipe diuji oleh komunitas di Depok untuk mengevaluasi efektivitas fitur yang dikembangkan. Hasil pengujian menunjukkan bahwa RENTALIN mampu meningkatkan keamanan, efisiensi manajemen kendaraan, serta mempermudah pencarian dan penyewaan motor \parencite{salsabilah2024rentalin}.

\subsection{\textit{Good Practices for Designing a UI/UX Motorcycle Display: A Systematic Literature Review}}
\label{subbab:ii.5.3}

\textcite{lehmann2024good} melakukan studi literatur untuk mengidentifikasi praktik terbaik dalam perancangan \textit{UI/UX} pengguna untuk tampilan sepeda motor. Penelitian ini dilakukan karena meningkatnya integrasi fitur digital pada kendaraan modern, yang membutuhkan antarmuka yang informatif, aman, mudah dipahami, dan mendukung kebutuhan pengendara dalam berbagai kondisi. Hasil analisis menunjukkan bahwa sebagian besar penelitian berfokus pada bagaimana antarmuka dapat memberikan informasi secara jelas dan tidak membebani pengendara, termasuk melalui pemilihan metode \textit{input}, pengaturan tampilan informasi, penggunaan warna, dan penggunaan elemen pendukung seperti ikon dan teks.

Penelitian ini juga mengidentifikasi berbagai rekomendasi desain yang sering muncul dalam penelitian terkait tampilan sepeda motor dan kendaraan secara umum \parencite{lehmann2024good}. Beberapa di antaranya seperti pemilihan metode \textit{input} yang sesuai dengan konteks penggunaan, misalnya penggunaan tombol untuk interaksi sederhana dan integrasi dengan perintah suara untuk tugas yang lebih kompleks. Selain itu, penyajian informasi yang sederhana dan mudah dipahami penting untuk menjaga fokus pengendara, termasuk dengan mengelompokkan elemen visual, menggunakan warna untuk membedakan prioritas informasi, serta menambahkan teks pada ikon untuk mengurangi kebingungan. Penelitian juga menekankan pentingnya \textit{feedback} yang jelas, personalisasi antarmuka, serta dukungan terhadap berbagai kondisi pengguna, seperti kelelahan atau perbedaan preferensi visual.

\subsection{\textit{User Expectation of Public Transport Design Experience for Electric Bike Sharing in Indonesia}}
\label{subbab:ii.5.4}

Penelitian yang dilakukan oleh \textcite{syabana2021user} bertujuan untuk mengetahui ekspektasi pengguna terhadap pengalaman desain layanan transportasi \textit{electric bike sharing} (EBS) di Indonesia. Penelitian ini menggunakan metode \textit{focus group interview} (FGI) dengan sepuluh partisipan yang berasal dari Indonesia dan berdomisili di Daegu, Korea Selatan. Melalui diskusi kelompok, penelitian ini menemukan sejumlah kendala yang sering ditemui dalam penggunaan transportasi publik di Indonesia, seperti ketidaknyamanan, kurangnya integrasi antarmoda, waktu perjalanan yang lama, serta infrastruktur yang kurang memadai. Selain itu, penelitian ini juga menjelaskan pengaruh faktor internal seperti pengalaman masa lalu, preferensi pribadi, dan motivasi. Serta faktor eksternal seperti kondisi lingkungan, pilihan transportasi yang tersedia, dan harga yang membentuk ekspektasi pengguna terhadap layanan EBS \parencite{syabana2021user}.

Berdasarkan hal tersebut, penelitian ini menyusun \textit{user adaptation stages} untuk menggambarkan proses perubahan perilaku pengguna dari praadopsi hingga penerimaan layanan. Penelitian ini juga menghasilkan \textit{journey expectation map} yang memvisualisasikan alur pengalaman pengguna yang diharapkan mulai dari tahap pencarian informasi, proses penyewaan, penggunaan sepeda listrik, hingga pengembalian, dan evaluasi layanan. Tidak hanya itu, kebutuhan pengguna yang mencakup aspek keamanan, kemudahan akses, keandalan sistem, efisiensi waktu tempuh, serta integrasi layanan dalam ekosistem transportasi publik yang lebih luas juga dijelaskan. Secara keseluruhan, \textcite{syabana2021user} menjelaskan bahwa pengembangan layanan EBS di Indonesia perlu mempertimbangkan peningkatan infrastruktur dan penyusunan regulasi yang mendukung agar layanan dapat memberikan pengalaman penggunaan yang lebih optimal.

\subsection{Perancangan Pengalaman Pengguna Aplikasi Sewa Sepeda Motor menggunakan Metode \textit{Human-Centered Design} (Studi Kasus: Rent-A-Motor Malang)}
\label{subbab:ii.5.5}

\textcite{muzakky2023perancangan} melakukan penelitian mengenai perancangan pengalaman pengguna untuk aplikasi sewa sepeda motor pada studi kasus Rent-A-Motor Malang. Penelitian ini dilatarbelakangi oleh permasalahan dalam proses penyewaan yang masih mengandalkan \textit{WhatsApp} sebagai media utama komunikasi, sehingga menimbulkan berbagai kendala seperti penyampaian informasi yang tidak terstruktur, proses pemesanan yang lama, dan ketidakefisienan dalam menangani banyak permintaan secara bersamaan. Melalui wawancara dengan penyewa dan pegawai, peneliti mengidentifikasikan kebutuhan utama pengguna, antara lain kemudahan memperoleh informasi ketersediaan motor, kemudahan pemesanan, transparansi status penyewaan, serta dukungan komunikasi yang lebih efektif \parencite{muzakky2023perancangan}. Kebutuhan tersebut menjadi dasar untuk merumuskan kebutuhan fungsional aplikasi, baik untuk \textit{guest} dan \textit{user}, maupun \textit{administrator}.

Penelitian ini menggunakan metode \textit{Human-Centered Design} (HCD) untuk menghasilkan desain aplikasi yang sesuai dengan kebutuhan dan konteks penggunaan. Tahapan yang dilakukan meliputi pembuatan \textit{persona}, penyusunan \textit{storyboard}, perancangan \textit{user flow}, \textit{information architecture}, \textit{wireframe}, \textit{mockup}, hingga pembuatan \textit{high-fidelity prototype}. Evaluasi desain dilakukan melalui \textit{usability testing} dan kuesioner \textit{System Usability Scale} (SUS) kepada total 12 responden yang mewakili tiga jenis aktor pengguna. Berdasarkan hasil pengujian, desain aplikasi memperoleh nilai efektivitas sebesar 89\%, efisiensi sebesar 0,36 \textit{goals}/detik, serta skor kepuasan SUS sebesar 76,4 yang termasuk kategori \textit{acceptable} dengan \textit{grade} B. Hasil ini menunjukkan bahwa desain aplikasi dapat digunakan secara efektif, relatif efisien, dan memberikan tingkat kepuasan pengguna yang baik. Penelitian ini juga memberikan rekomendasi perbaikan seperti penambahan fitur notifikasi, halaman panduan pemesanan, serta filter pada riwayat pesanan untuk meningkatkan pengalaman pengguna \parencite{muzakky2023perancangan}.